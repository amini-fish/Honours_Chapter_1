% Options for packages loaded elsewhere
\PassOptionsToPackage{unicode}{hyperref}
\PassOptionsToPackage{hyphens}{url}
%
\documentclass[
  a4paper]{article}
\usepackage{amsmath,amssymb}
\usepackage{iftex}
\ifPDFTeX
  \usepackage[T1]{fontenc}
  \usepackage[utf8]{inputenc}
  \usepackage{textcomp} % provide euro and other symbols
\else % if luatex or xetex
  \usepackage{unicode-math} % this also loads fontspec
  \defaultfontfeatures{Scale=MatchLowercase}
  \defaultfontfeatures[\rmfamily]{Ligatures=TeX,Scale=1}
\fi
\usepackage{lmodern}
\ifPDFTeX\else
  % xetex/luatex font selection
\fi
% Use upquote if available, for straight quotes in verbatim environments
\IfFileExists{upquote.sty}{\usepackage{upquote}}{}
\IfFileExists{microtype.sty}{% use microtype if available
  \usepackage[]{microtype}
  \UseMicrotypeSet[protrusion]{basicmath} % disable protrusion for tt fonts
}{}
\makeatletter
\@ifundefined{KOMAClassName}{% if non-KOMA class
  \IfFileExists{parskip.sty}{%
    \usepackage{parskip}
  }{% else
    \setlength{\parindent}{0pt}
    \setlength{\parskip}{6pt plus 2pt minus 1pt}}
}{% if KOMA class
  \KOMAoptions{parskip=half}}
\makeatother
\usepackage{xcolor}
\usepackage[margin=1in]{geometry}
\usepackage{graphicx}
\makeatletter
\newsavebox\pandoc@box
\newcommand*\pandocbounded[1]{% scales image to fit in text height/width
  \sbox\pandoc@box{#1}%
  \Gscale@div\@tempa{\textheight}{\dimexpr\ht\pandoc@box+\dp\pandoc@box\relax}%
  \Gscale@div\@tempb{\linewidth}{\wd\pandoc@box}%
  \ifdim\@tempb\p@<\@tempa\p@\let\@tempa\@tempb\fi% select the smaller of both
  \ifdim\@tempa\p@<\p@\scalebox{\@tempa}{\usebox\pandoc@box}%
  \else\usebox{\pandoc@box}%
  \fi%
}
% Set default figure placement to htbp
\def\fps@figure{htbp}
\makeatother
\setlength{\emergencystretch}{3em} % prevent overfull lines
\providecommand{\tightlist}{%
  \setlength{\itemsep}{0pt}\setlength{\parskip}{0pt}}
\setcounter{secnumdepth}{5}
\usepackage{booktabs}
\usepackage{longtable}
\usepackage{array}
\usepackage{multirow}
\usepackage{wrapfig}
\usepackage{float}
\usepackage{colortbl}
\usepackage{pdflscape}
\usepackage{tabu}
\usepackage{threeparttable}
\usepackage{threeparttablex}
\usepackage[normalem]{ulem}
\usepackage{makecell}
\usepackage{xcolor}
\usepackage{bookmark}
\IfFileExists{xurl.sty}{\usepackage{xurl}}{} % add URL line breaks if available
\urlstyle{same}
\hypersetup{
  pdftitle={Supplemental Information for:},
  hidelinks,
  pdfcreator={LaTeX via pandoc}}

\title{Supplemental Information for:}
\usepackage{etoolbox}
\makeatletter
\providecommand{\subtitle}[1]{% add subtitle to \maketitle
  \apptocmd{\@title}{\par {\large #1 \par}}{}{}
}
\makeatother
\subtitle{A review of the utility and application of relatedness and
kinship in elasmobranchs}
\author{Samuel R. Amini*, Nicole M. Phillips, Pierre Feutry, Peter M.
Kyne\\
\strut \\
* \emph{Author produced this Rmarkdown}}
\date{Apil 2025}

\begin{document}
\maketitle

{
\setcounter{tocdepth}{5}
\tableofcontents
}
\begin{center}\rule{0.5\linewidth}{0.5pt}\end{center}

\pagebreak

\subsection{Supplementary Figure 1}\label{supplementary-figure-1}

\includegraphics[width=1\linewidth]{Chapter_1_Supplementary_Materials_files/figure-latex/Supplementary Figure 1-1}

Figure S1. A) The IUCN Status of species in each study examining
relatedness in elasmobranchs. B) The IUCN status of each species,
grouped by study category. In clockwise order starting from the top left
these are: reproductive behaviour, population genetics, sociality, and
demography.

\begin{center}\rule{0.5\linewidth}{0.5pt}\end{center}

\pagebreak

\subsection{Supplementary Figure 2}\label{supplementary-figure-2}

\includegraphics[width=1\linewidth]{Chapter_1_Supplementary_Materials_files/figure-latex/Supplementary Figure 2-1}

Figure S2. A heatmap of estimators used across all studies, visualised
as pairwise combinations. Self-comparisons represent the number of times
an estimator was used in isolation.

\begin{center}\rule{0.5\linewidth}{0.5pt}\end{center}

\pagebreak

\subsection{Supplementary Figure 3}\label{supplementary-figure-3}

\includegraphics[width=1\linewidth]{Chapter_1_Supplementary_Materials_files/figure-latex/Supplementary Figure 3-1}

Figure S3. A boxplot of the number of samples used in each research
category, where the box denotes the inter-quartile range, and solid line
denotes the median, and dots represent outliers.

\end{document}
